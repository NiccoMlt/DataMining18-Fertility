% !TeX root = ./report.tex
% !TeX encoding = UTF-8 Unicode
% !TeX spellcheck = it_IT
% !TeX program = arara
% !TeX options = --log --verbose --language=it "%DOC%"

% arara: lualatex:      { interaction: batchmode }
% arara: lualatex:      { interaction: nonstopmode, synctex: yes }

\documentclass[a4paper,11pt,twoside,notitlepage,final]{scrartcl}

\usepackage{fancyvrb}

\begin{VerbatimOut}{\jobname.xmpdata}
\Title{Fertility Dataset Analysis}
\Author{Niccolò Maltoni}
\Copyright{Questo documento è fornito sotto licenza Creative Commons Attribution-ShareAlike 4.0 International}
\CopyrightURL{http://creativecommons.org/licenses/by-sa/4.0}
\end{VerbatimOut}

\usepackage[english,italian]{varioref}

\usepackage[luatex,dvipsnames,table,xcdraw]{xcolor}
\usepackage[a-1b]{pdfx}

%% Font
\usepackage{fontspec}
\defaultfontfeatures{Ligatures=TeX}
\setmainfont[SmallCapsFont={* Caps}]{Latin Modern Roman}
\setsansfont{Latin Modern Sans}

%% Matematica
\usepackage{amsmath}
\usepackage[math-style=ISO]{unicode-math}
\setmathfont{Latin Modern Math}
\usepackage[output-decimal-marker={,},binary-units]{siunitx}

\defaultfontfeatures{} % reset per mono font
\setmonofont{Latin Modern Mono}

%% Lingue
\usepackage[strict=true,autostyle=true,english=american,italian=guillemets]{csquotes}
\usepackage{polyglossia}
\setmainlanguage[babelshorthands]{italian}
\setotherlanguage[variant=american]{english}

%% Altri pacchetti
\usepackage{graphicx}
\graphicspath{{fig}}
\usepackage{xargs}
\usepackage[colorinlistoftodos,prependcaption,textsize=tiny]{todonotes}
\newcommandx{\unsure}[2][1=]{\todo[linecolor=red,backgroundcolor=red!25,bordercolor=red,#1]{#2}}
\newcommandx{\change}[2][1=]{\todo[linecolor=blue,backgroundcolor=blue!25,bordercolor=blue,#1]{#2}}
\newcommandx{\info}[2][1=]{\todo[linecolor=OliveGreen,backgroundcolor=OliveGreen!25,bordercolor=OliveGreen,#1]{#2}}
\newcommandx{\improvement}[2][1=]{\todo[linecolor=Plum,backgroundcolor=Plum!25,bordercolor=Plum,#1]{#2}}
\newcommandx{\thiswillnotshow}[2][1=]{\todo[disable,#1]{#2}}
\usepackage{subcaption}
\usepackage{caption}
\usepackage{scrhack}
\usepackage{float}

\usepackage{geometry}
\geometry{a4paper,heightrounded}
\usepackage{setspace}
\onehalfspacing{}

% \usepackage[maxcitenames=2,mincitenames=2,maxbibnames=99,minbibnames=99,style=ieee,giveninits=true,backend=biber]{biblatex}
% \addbibresource{biblio.bib}

% \usepackage[htt]{hyphenat}
% \usepackage{enumerate}

\usepackage{xurl}
\usepackage{microtype}

\hypersetup{%
  pdfpagemode={UseNone},
  hidelinks,
  hypertexnames=false,
  linktoc=all,
  unicode=true,
  pdftoolbar=false,
  pdfmenubar=false,
  plainpages=false,
  breaklinks,
  pdfstartview={Fit},
  pdflang={it}
}

\usepackage[english,italian,nameinlink]{cleveref}

\title{\LARGE{\textbf{Fertility Dataset Analysis}}}

\author{Niccolò~Maltoni}

\date{%
  \small{Data Mining}\\%
  \small{Anno accademico 2019--2020}
}

\begin{document}

\maketitle

\section{Introduzione}\label{sec:intro}

Per la produzione dell'elaborato, si è scelto di condurre le operazioni di studio e mining dei dati sul dataset chiamato \emph{Fertility Data Set}\footnote{\url{http://archive.ics.uci.edu/ml/datasets/Fertility}}, prelevato dal sito \emph{UCI Machine Learning}.
In questa \nameCref{sec:intro} verranno presentati il dataset sul quale è stato svolto il lavoro.
Verrà preso in considerazione il dominio di provenienza del dataset e la struttura dei dati in esso contenuti prima di apportarne delle modifiche,
per poter essere in grado di dedurre conoscenza dai dati in questione.

\subsection{Descrizione del dominio}

Il dataset contiene i risultati degli esami di fertilità effettuati su campioni di seme secondo i criteri dell'Organizzazione Mondiale della Sanità (\emph{World Health Organization 2010});
l'analisi sul seme maschile permettono infatti di individuare potenziali problemi di sterilità nei pazienti.

La possibilità di alterazioni del seme possono essere correlate ad aspetti diversi della vita dell'individuo:
oltre alla genetica, pare che la sterilità maschile possa essere correlata anche a fattori socio-demografici, ambientali e di stile di vita.

Può essere interessante analizzare quanto effettivamente la presenza di malattie e traumi non necessariamente locali all'apparato genitale possano incidere sulla fertilità del paziente,
come anche uno stile di vita particolarmente sedentario o il consumo abituale di sostanze non ritenute particolarmente salutari come tabacco da fumo e alcool.
In questo modo, tramite Data Mining sarebbe possibile aiutare la diagnosi medica tramite il supporto informatico.

% Fertility rates have dramatically decreased in the last two decades, especially in men. It has been described that environmental factors, as well as life habits, may affect semen quality.
% Artificial intelligence techniques are now an emerging methodology as decision support systems in medicine.
% In this paper we compare three artificial intelligence techniques, decision trees, Multilayer Perceptron and Support Vector Machines,
% in order to evaluate their performance in the prediction of the seminal quality from the data of the environmental factors and lifestyle.
% To do that we collect data by a normalized questionnaire from young healthy volunteers and then, we use the results of a semen analysis to asses the accuracy in the prediction of the three classification methods mentioned above.
% The results show that Multilayer Perceptron and Support Vector Machines show the highest accuracy, with prediction accuracy values of 86\% for some of the seminal parameters.
% In contrast decision trees provide a visual and illustrative approach that can compensate the slightly lower accuracy obtained.
% In conclusion artificial intelligence methods are a useful tool in order to predict the seminal profile of an individual from the environmental factors and life habits.
% From the studied methods, Multilayer Perceptron and Support Vector Machines are the most accurate in the prediction.
% Therefore these tools, together with the visual help that decision trees offer, are the suggested methods to be included in the evaluation of the infertile patient.

\subsection{Descrizione del dataset}

Questo dataset contiene i risultati degli esami di fertilità di 100 uomini;
il risultato di questi esami può essere di due tipologie, Normale oppure Alterato, che indica un potenziale problema di sterilità nel paziente.
Questi risultati sono correlati all'interno del dataset con altri dati di natura socio-demografica, ambientale e abitudinaria dei soggetti analizzati.

Il file \href{http://archive.ics.uci.edu/ml/machine-learning-databases/00244/fertility_Diagnosis.txt}{\texttt{fertility\_Diagnosis.txt}}
contiene l'intero dataset, costituito da linee costituenti i record, con gli attributi separati da virgole.
Al suo interno, sono presenti \emph{100 istanze} composte di \emph{10 attributi};
di seguito sono riportati con una breve descrizione.

\begin{description}
  \item[Stagione]
    Attributo nominale che rappresenta la stagione in cui è stato svolto l'esame.
    Può assumere quattro possibili valori, corrispondenti alle stagioni astronomiche dell'anno solare:
    \begin{itemize}
      \item \texttt{-1}: \emph{primavera};
      \item \texttt{-0.33}: \emph{estate};
      \item \texttt{0.33}: \emph{autunno};
      \item \texttt{1}: \emph{inverno}.
    \end{itemize}
    % \begin{figure}[H]
    %   \centering
    %   \missingfigure[figwidth=0.5\textwidth]{Distribuzione dei valori della stagione}
    %   \caption{Media: \(x\) Deviazione Standard: \(y\)}%
    %   \label{fig:season}
    % \end{figure}
  \item[Età]
    Attributo che rappresenta l'età dei partecipanti al test al momento dell'analisi.
    Può assumere valori numerici in un range compreso tra \texttt{0.5} (corrispondente a 18 anni) e \texttt{1} (corrispondente a 36 anni).
    % \begin{figure}[H]
    %   \centering
    %   \missingfigure[figwidth=0.5\textwidth]{Distribuzione dei valori dell'età}
    %   \caption{Media: \(x\) Deviazione Standard: \(y\)}%
    %   \label{fig:age}
    % \end{figure}
  \item[Malattie infantili]
    Attributo binario che rappresenta la presenza o meno di patologie clinicamente rilevanti (\emph{Varicella}, \emph{Orecchioni}, \emph{Morbillo}, etc.) contratte dal paziente durante l'infanzia.
    % \unsure{Davvero sono a rovescio!?}
    Può assumere i valori \texttt{0} (sì) e \texttt{1} (no).
    % \begin{figure}[H]
    %   \centering
    %   \missingfigure[figwidth=0.5\textwidth]{Distribuzione dei valori delle malattie}
    %   \caption{Media: \(x\) Deviazione Standard: \(y\)}%
    %   \label{fig:disease}
    % \end{figure}
  \item[Incidenti o traumi seri]
    Attributo binario che rappresenta la presenza o meno di traumi/incidenti rilevanti nella storia clinica del paziente.
    % \unsure{Davvero sono a rovescio!?}
    Può assumere i valori \texttt{0} (sì) e \texttt{1} (no).
    % \begin{figure}[H]
    %   \centering
    %   \missingfigure[figwidth=0.5\textwidth]{Distribuzione dei valori dei traumi}
    %   \caption{Media: \(x\) Deviazione Standard: \(y\)}%
    %   \label{fig:trauma}
    % \end{figure}
  \item[Interventi chirurgici]
    Attributo binario che rappresenta la presenza o meno di interventi chirurgici nella storia clinica del paziente.
    % \unsure{Davvero sono a rovescio!?}
    Può assumere i valori \texttt{0} (sì) e \texttt{1} (no).
    % \begin{figure}[H]
    %   \centering
    %   \missingfigure[figwidth=0.5\textwidth]{Distribuzione dei valori delle operazioni chirurgiche}
    %   \caption{Media: \(x\) Deviazione Standard: \(y\)}%
    %   \label{fig:surgical}
    % \end{figure}
  \item[Febbre]
    Attributo nominale che rappresenta l'aver contratto, da parte del candidato, febbri particolarmente alte nell'ultimo anno:
    Può assumere tre possibili valori, corrispondenti al periodo:
    \begin{itemize}
      \item \texttt{-1}: \emph{meno di tre mesi} dalla data dell'analisi;
      \item \texttt{0}: \emph{più di tre mesi} dalla data dell'analisi;
      \item \texttt{1}: \emph{mai nell'ultimo anno}.
    \end{itemize}
    % \begin{figure}[H]
    %   \centering
    %   \missingfigure[figwidth=0.5\textwidth]{Distribuzione dei valori della febbre}
    %   \caption{Media: \(x\) Deviazione Standard: \(y\)}%
    %   \label{fig:fever}
    % \end{figure}
  \item[Consumo di alcool]
    Attributo nominale che rappresenta la frequenza del consumo di alcool, del soggetto in analisi.
    Può assumere cinque valori:
    \begin{itemize}
      \item \texttt{0.2}: \emph{più volte al giorno};
      \item \texttt{0.4}: \emph{una volta al giorno};
      \item \texttt{0.6}: \emph{più volte alla settimana};
      \item \texttt{0.8}: \emph{una volta a settimana};
      \item \texttt{1}: \emph{molto raramente o mai}.
    \end{itemize}
    % \begin{figure}[H]
    %   \centering
    %   \missingfigure[figwidth=0.5\textwidth]{Distribuzione dei valori dell'alcool}
    %   \caption{Media: \(x\) Deviazione Standard: \(y\)}%
    %   \label{fig:alcool}
    % \end{figure}
  \item[Fumo]
    Attributo nominale che rappresenta le abitudini del soggetto rispetto al fumo.
    Può assumere tre valori:
    \begin{itemize}
      \item \texttt{-1}: \emph{non fumatore};
      \item \texttt{0}: \emph{fumatore sporadico};
      \item \texttt{1}: \emph{fumatore abituale}.
    \end{itemize}
    % \begin{figure}[H]
    %   \centering
    %   \missingfigure[figwidth=0.5\textwidth]{Distribuzione dei valori del fumo}
    %   \caption{Media: \(x\) Deviazione Standard: \(y\)}%
    %   \label{fig:smoke}
    % \end{figure}
  \item[Ore spese seduto]
    Attributo che quantifica le ore trascorse seduto al giorno dal soggetto.
    Può assumere valori numerici in un range compreso tra \texttt{0.06} (corrispondente a un'ora) e \texttt{1} (16 ore).
    % \begin{figure}[H]
    %   \centering
    %   \missingfigure[figwidth=0.5\textwidth]{Distribuzione dei valori delle ore passate seduto}
    %   \caption{Media: \(x\) Deviazione Standard: \(y\)}%
    %   \label{fig:sitting}
    % \end{figure}
  \end{description}

Oltre agli attributi sopra riportati, i record del dataset sono etichettati con un attributo classe corrispondente alla diagnosi:
il carattere \texttt{N} corrisponde ad una situazione clinica \emph{normale}, mentre \texttt{O} corrisponde ad una diagnosi \emph{alterata}.

% Il problema delineato è dunque affrontabile come un quesito di classificazione binaria:
% si deve infatti poter costruire un modello che, dati certi parametri per gli attributi presi in causa, permetta di prevedere un potenziale problema di fertilità o meno.

\section{Grafici}

\end{document}
